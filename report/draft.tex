
\autoref{Rice:algo:sample} demonstrates the formatting of pseudo code. 
Please carefully check the source files and learn how to use this style. 
Importantly:

\begin{itemize}
\item Always state your input.

\item State the output if any. 

\item Always number your lines for quick referral.

\item Always declare and initialize your local variables. 

\item Always use \CMD{\gets} (``$\gets$'') for assignments.
%Always use \textbackslash gets for assignments.
\end{itemize}

\begin{algorithm}[!t]
\AlgoFontSize
\DontPrintSemicolon

\KwGlobal{max. calories of daily intake $\mathcal{C}$}
\KwGlobal{calories per bowl of rice $\mathcal{B}$}
\BlankLine

\SetKwFunction{fInferExp}{InferExp}
\SetKwFunction{fDoExercise}{DoExercise}

\KwIn{Proof context $\mathcal{E}=(\Sigma,\Gamma)$, program expression $e$ and pre-condition $\Delta$}
\KwOut{A set of (context, return name, post-condition) $\left\{ \mathcal{E'},\mathtt{r},\Delta' \right\}$}
\Proc{\fInferExp{$\mathcal{E},e,\Delta$}}{
  Apply \nameref{rule:fv-exists} and \nameref{rule:fv-spec} to normalize $\Delta$ and update $\mathcal{E}$\;
  \Switch{the structure of $e$}{
     \uCase{identifier $v$}{ 
        Apply \nameref{rule:fv-var}, \nameref{rule:fv-fun-eval} if applicable and get $\mathcal{E}', \mathtt{res}, \Delta'$\;
        \Return{ $\left\{ (\mathcal{E}'(\leftarrow\mathtt{res}), \mathtt{res}, \Delta') \right\}$  }\;
     }
     \uCase{constant $k$}{ \Return{ $\left\{ (\mathcal{E}', \mathtt{res}, \Delta') \right\}$} obtained by \nameref{rule:fv-cst} }
     \Case{last value}{
        do this\;
        break\; 
    }
     \Other{
       for the other values\;
       do that\;
    } 
  }
  \uIf{$cal \geq \mathcal{C}$}{
    \Return $\mathcal{C}$\;
  }
  \Else{
    \Return $cal - \fDoExercise{n}$\;
  }
}

\BlankLine

\KwIn{time duration (in minutes) of exercise $t$}
\KwOut{calories consumed}
\Func{\fDoExercise{$t$}}{
  apply \ref{rule:fv-exists} \;
  \lFor{$i \gets 1$ \To $t$}{$cal \gets cal + i$}
  \Return $cal$\;
}

\caption{Forward verification implementation}
\label{Rice:algo:sample}
\end{algorithm}
